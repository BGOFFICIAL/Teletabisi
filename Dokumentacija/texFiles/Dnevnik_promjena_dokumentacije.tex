\chapter{Dnevnik promjena dokumentacije}
		
		\textbf{\textit{Kontinuirano osvježavanje}}\\
				
		
		\begin{longtblr}[
				label=none
			]{
				width = \textwidth, 
				colspec={|X[2]|X[13]|X[3]|X[3.2]|}, 
				rowhead = 1
			}
			\hline
			\textbf{Rev.}	& \textbf{Opis promjene/dodatka} & \textbf{Autori} & \textbf{Datum}\\[3pt] \hline
			0.1 & Napravljen predložak & Neven Pralas & 30.10.2023. 		\\[3pt] \hline 
			0.2	& Napisan opis projektnog zadatka & Neven Pralas & 02.11.2023. 	\\[3pt] \hline 
			0.3 & Opisani funkcionalni zahtjevi i svi aktori sustava & Neven Pralas & 03.11.2023. \\[3pt] \hline 
			0.4 & Dodan prvi dio obrazaca uporabe & Neven Pralas & 05.11.2023. \\[3pt] \hline 
			0.4.1 & Dodan preostali dio obrazaca uporabe & Neven Pralas & 06.11.2023. \\[3pt] \hline 
			0.5 & Dodani sekvencijski dijagrami i ostali zahtjevi & Tin \newline Ogrizek & 09.11.2023. \\[3pt] \hline 
			0.5.1 & Popravljeni sekvencijski dijagrami & Tin \newline Ogrizek & 12.11.2023. \\[3pt] \hline 
			0.6 & Izmijenjeni početni dijelovi dokumentacije & Neven Pralas & 13.11.2023. \\[3pt] \hline 
			0.7 & Opisana arhitektura računala & Filip \newline Posavec i Neven Pralas & 14.11.2023. \\[3pt] \hline 
			0.8 & Opisana baza podataka i dijagrami baze podataka & Bruno Petković & 15.11.2023. \\[3pt] \hline 
			1.0 & Dodani dijagrami razreda\newline Verzija samo s bitnim dijelovima za 1. ciklus & Tin \newline Ogrizek & 17.11.2023. \\[3pt] \hline 
			1.1 & Uređivanje teksta -- funkcionalni i nefunkcionalni zahtjevi & * \newline * & 14.09.2013. \\[3pt] \hline 
			1.2 & Manje izmjene:Timer - Brojilo vremena & * & 15.09.2013. \\[3pt] \hline 
			1.3 & Popravljeni dijagrami obrazaca uporabe & * & 15.09.2013. \\[3pt] \hline 
			1.5 & Generalna revizija strukture dokumenta & * & 19.09.2013. \\[3pt] \hline 
			1.5.1 & Manja revizija (dijagram razmještaja) & * & 20.09.2013. \\[3pt] \hline 
			\textbf{2.0} & Konačni tekst predloška dokumentacije  & * & 28.09.2013. \\[3pt] \hline 
			&  &  & \\[3pt] \hline	
		\end{longtblr}
	
	
		\textit{Moraju postojati glavne revizije dokumenata 1.0 i 2.0 na kraju prvog i drugog ciklusa. Između tih revizija mogu postojati manje revizije već prema tome kako se dokument bude nadopunjavao. Očekuje se da nakon svake značajnije promjene (dodatka, izmjene, uklanjanja dijelova teksta i popratnih grafičkih sadržaja) dokumenta se to zabilježi kao revizija. Npr., revizije unutar prvog ciklusa će imati oznake 0.1, 0.2, …, 0.9, 0.10, 0.11.. sve do konačne revizije prvog ciklusa 1.0. U drugom ciklusu se nastavlja s revizijama 1.1, 1.2, itd.}