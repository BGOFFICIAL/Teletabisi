\chapter{Arhitektura i dizajn sustava}

		Arhitekturu sustava općenito (pa tako i našeg) možemo podijeliti na tri dijela:
	\begin{itemize}
		\item 	\textit{Web poslužitelj}
		\item 	\textit{Web aplikacija}
		\item 	\textit{Baza podataka}		
	\end{itemize}
	
	\textbf{Internetski preglednik (web preglednik, web browser)} je dio arhitekture koji korisniku omogućuje pregled web-stranica pa tako i samoga sadržaja koji se nalazi na njoj. Korisnik putem web preglednika šalje HTTP zahtjev poslužitelju za dohvat željenog sadržaja i čeka HTTP odgovor. HTTP je protokol bez stanja što znači da primatelj ne smije zadržati stanje sesije iz prethodnih zahtjeva. Neki od popularnijih web preglednika su "Google Chrome" ili "Opera".
	
	\textbf{Web poslužitelj} je osnova svake web aplikacije. On šalje klijentu HTTP odgovor na određeni HTTP zahtjev. Korisnik koristi web aplikaciju na način da web aplikacija obrađuje zahtjev te ovisno o potrebi pristupa bazi podataka. Poslužitelj vraća odgovor u obliku HTML dokumenta koji je vidljiv korisniku.
		
    \begin{figure}[H]
    	\includegraphics[scale=1.2]{slike/HTTP.PNG} %veličina slike u odnosu na originalnu datoteku i pozicija slike
    	\centering
    	\caption{Prikaz HTTP komunikacije između klijenta i poslužitelja}
    	\label{fig:promjene}
    \end{figure}
    
    \textit{\newline \newline}
    
    Naša grupa je za projekt na predmetu Programsko inženjerstvo (ak.god. 2023./2024.) odabrala React sustav za potrebe frontenda. Odabrali smo React zbog njegove jednostavne i moćne arhitekture koja omogućuje brz i održiv razvoj web aplikacija. React se temelji na principu komponenata, što znači da aplikaciju gradimo kao skup neovisnih dijelova sučelja, svaki s vlastitom logikom i stilovima. Time svaka komponenta postaje ponovno upotrebljiva i lako zamjenjiva što olakšava razvoj aplikacije.
    
    Za backend smo odabrali raditi u programskom jeziku Java i to u Spring Boot-u. Koristili smo Maven alat za izgradnju programskog koda. Spring Boot nam se svidio zbog dvije karakteristike: inverzija kontrole gdje sam Spring Boot kontrolira izvršavanje programskog koda te injektiranje objekata o kojima ovisi rad koda. Neke od karakteristika Spring Boot-a su: unaprijed pripremljene funkcionalnosti, nema generiranja klasa i koda već se koriste unaprijed definirane biblioteke. U svakom slučaju Spring Boot zna dosta olakšati programeru posao. Također olakšava i spajanje na bazu podataka o kojoj će biti više riječi u sljedećoj cjelini.
    
    Razvojno okruženje koje smo koristili zavisilo je od člana do člana ekipe, neki su radili u Eclipse-u, neki u IntelliJ. Koristili smo GitHub sustav za upravljanje verzijama programske potpore te TeXstudio za pisanje dokumentacije.
    
    \begin{figure}[H]
    	\includegraphics[scale=0.5]{slike/SpringBoot.PNG} %veličina slike u odnosu na originalnu datoteku i pozicija slike
    	\centering
    	\caption{Neke od koristi Spring Boot-a}
    	\label{fig:promjene}
    \end{figure}
		

				
		\section{Baza podataka}
			
			\textbf{\textit{dio 1. revizije}}\\
			
		\textit{Potrebno je opisati koju vrstu i implementaciju baze podataka ste odabrali, glavne komponente od kojih se sastoji i slično.}
		
			\subsection{Opis tablica}
			

				\textit{Svaku tablicu je potrebno opisati po zadanom predlošku. Lijevo se nalazi točno ime varijable u bazi podataka, u sredini se nalazi tip podataka, a desno se nalazi opis varijable. Svjetlozelenom bojom označite primarni ključ. Svjetlo plavom označite strani ključ}
				
				
				\begin{longtblr}[
					label=none,
					entry=none
					]{
						width = \textwidth,
						colspec={|X[6,l]|X[6, l]|X[20, l]|}, 
						rowhead = 1,
					} %definicija širine tablice, širine stupaca, poravnanje i broja redaka naslova tablice
					\hline \SetCell[c=3]{c}{\textbf{korisnik - ime tablice}}	 \\ \hline[3pt]
					\SetCell{LightGreen}IDKorisnik & INT	&  	Lorem ipsum dolor sit amet, consectetur adipiscing elit, sed do eiusmod  	\\ \hline
					korisnickoIme	& VARCHAR &   	\\ \hline 
					email & VARCHAR &   \\ \hline 
					ime & VARCHAR	&  		\\ \hline 
					\SetCell{LightBlue} primjer	& VARCHAR &   	\\ \hline 
				\end{longtblr}
				
				
			
			\subsection{Dijagram baze podataka}
				\textit{ U ovom potpoglavlju potrebno je umetnuti dijagram baze podataka. Primarni i strani ključevi moraju biti označeni, a tablice povezane. Bazu podataka je potrebno normalizirati. Podsjetite se kolegija "Baze podataka".}
			
			\eject
			
			
		\section{Dijagram razreda}
		
			\textit{Potrebno je priložiti dijagram razreda s pripadajućim opisom. Zbog preglednosti je moguće dijagram razlomiti na više njih, ali moraju biti grupirani prema sličnim razinama apstrakcije i srodnim funkcionalnostima.}\\
			
			\textbf{\textit{dio 1. revizije}}\\
			
			\textit{Prilikom prve predaje projekta, potrebno je priložiti potpuno razrađen dijagram razreda vezan uz \textbf{generičku funkcionalnost} sustava. Ostale funkcionalnosti trebaju biti idejno razrađene u dijagramu sa sljedećim komponentama: nazivi razreda, nazivi metoda i vrste pristupa metodama (npr. javni, zaštićeni), nazivi atributa razreda, veze i odnosi između razreda.}\\
			
			\textbf{\textit{dio 2. revizije}}\\			
			
			\textit{Prilikom druge predaje projekta dijagram razreda i opisi moraju odgovarati stvarnom stanju implementacije}
			
			
			
			\eject
		
		\section{Dijagram stanja}
			
			
			\textbf{\textit{dio 2. revizije}}\\
			
			\textit{Potrebno je priložiti dijagram stanja i opisati ga. Dovoljan je jedan dijagram stanja koji prikazuje \textbf{značajan dio funkcionalnosti} sustava. Na primjer, stanja korisničkog sučelja i tijek korištenja neke ključne funkcionalnosti jesu značajan dio sustava, a registracija i prijava nisu. }
			
			
			\eject 
		
		\section{Dijagram aktivnosti}
			
			\textbf{\textit{dio 2. revizije}}\\
			
			 \textit{Potrebno je priložiti dijagram aktivnosti s pripadajućim opisom. Dijagram aktivnosti treba prikazivati značajan dio sustava.}
			
			\eject
		\section{Dijagram komponenti}
		
			\textbf{\textit{dio 2. revizije}}\\
		
			 \textit{Potrebno je priložiti dijagram komponenti s pripadajućim opisom. Dijagram komponenti treba prikazivati strukturu cijele aplikacije.}