\chapter{Zaključak i budući rad}
		
		Projektni zadatak naše grupe je bio izrada  web aplikacije koja korisnicima omogućuje jednostavno naručivanje za fizikalnu terapiju i medicinsku rehabilitaciju. Razvoj projekta je trajao 13 tjedna te smo u tom periodu ostvarili gotovo sve funkcionalnosti, a projekt smo ostvarili u dvije faze. U prvoj fazi cilj nam je bio implementirati generičke funkcionalnosti poput sustava za prijavu u stranicu, dizajniranje baze podataka i izrada početne stranice.
		Započeli smo izradu projekta upoznavanjem i raspravom o tehnologijama koje koristimo. Podijelili smo se u dvije skupine, jednu koja je radila fronted i jednu koja
		je radila backend. Relativno rano smo postavili osnovni dizaj obojih aplikacija što se pokazalo kontraproduktivno zbog manjka iskustva kod većine članova koji nisu bili upoznati s tim arhitekturama.
		
		U drugoj fazi bilo je potrebno dovršiti preostale funkcionalnosti, krenuli smo dobrim tempom, ali se je ubrzo pokazalo kako je manjak praktičnog iskustva drastično usporio razvoj. Sporom razvoju je doprinio i veliki broj obaveza koje su svi članovi imali. Iako je razvoj usporio više iskusni članovi su preuzeli inicijativu i više vremena uložili u izradu aplikacije kako bi osigurali što veću funkcionalnost.


Radeći retrospektivu zaključujem kako možda nismo bili najfunkcionalniji i
najorganiziraniji tim, ali smo naučili puno kako se to u budućnosti ne bi ponovilo.
Uspješno smo riješili izazove koji su se pojavili pred nama i na samom kraju smo
isporučili funkcionalni proizvod kojim smo izrazito zadovoljni i ponosni.
\eject 
		\eject 