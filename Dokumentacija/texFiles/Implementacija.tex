\chapter{Implementacija i korisničko sučelje}
		
		
		\section{Korištene tehnologije i alati}
		
			
			 {Komunikacija u timu se obavljala pomoću aplikacija \underbar{WhatsApp}\footnote{Aplikacija za besplatnu razmjenu poruka, fotografija, videozapisa i drugih datoteka, te uspostavljanje glasovnih i videopoziva putem mobilnog interneta pametnim telefonima:\newline\url{https://www.whatsapp.com/}} i \underbar{Discord}\footnote{Besplatna aplikacija za komunikaciju putem tekstualnih poruka i glasovnih i video poziva:\newline\url{https://discord.com/}}.
			 	\newline Dokumentacija je bila pisana pomoću alata TeXstudio i TexLive. Za izradu UML dijagrama korišten je alat \underbar{Astah Professional}\footnote{https://astah.net/products/astah-professional/}.Kao razvojno okruženje korišten je \underbar{IntelliJ IDEA Ultimate}\footnote{https://www.jetbrains.com/lp/intellij-frameworks/}. IntelliJ IDEA je integrirano razvojno okruženje napisano u Javi za razvoj računalnog softvera napisanog u Javi, Kotlinu, Groovyju i drugim jezicima koji se temelje na JVM-u. Razvio ga je JetBrains i dostupan je u dva izdanja, Ultimate i Community. Aplikacija je napisana koristeći radni okvir \underbar{Spring Boot}\footnote{\url{https://spring.io/projects/spring-boot}} i jezik \underbar{Java}\footnote{\url{https://www.java.com/en/}} za izradu backenda. Za izradu frontenda je koristen \underbar{React}\footnote{Besplatna JavaScript biblioteka otvorenog tipa za izgradnju korisničkih sučelja:\newline\url{https://reactjs.org/}}. Za izradu naše baze podataka koristili smo \underbar{PostgreSQL}\footnote{https://www.postgresql.org/}.}
			
			
			\eject 
		
	
		\section{Ispitivanje programskog rješenja}
			
			\textbf{\textit{dio 2. revizije}}\\
			
			 \textit{U ovom poglavlju je potrebno opisati provedbu ispitivanja implementiranih funkcionalnosti na razini komponenti i na razini cijelog sustava s prikazom odabranih ispitnih slučajeva. Studenti trebaju ispitati temeljnu funkcionalnost i rubne uvjete.}
	
			
			\subsection{Ispitivanje komponenti}
			\textit{Potrebno je provesti ispitivanje jedinica (engl. unit testing) nad razredima koji implementiraju temeljne funkcionalnosti. Razraditi \textbf{minimalno 6 ispitnih slučajeva} u kojima će se ispitati redovni slučajevi, rubni uvjeti te izazivanje pogreške (engl. exception throwing). Poželjno je stvoriti i ispitni slučaj koji koristi funkcionalnosti koje nisu implementirane. Potrebno je priložiti izvorni kôd svih ispitnih slučajeva te prikaz rezultata izvođenja ispita u razvojnom okruženju (prolaz/pad ispita). }
			
			
			
			\subsection{Ispitivanje sustava}
			
			 \textit{Potrebno je provesti i opisati ispitivanje sustava koristeći radni okvir Selenium\footnote{\url{https://www.seleniumhq.org/}}. Razraditi \textbf{minimalno 4 ispitna slučaja} u kojima će se ispitati redovni slučajevi, rubni uvjeti te poziv funkcionalnosti koja nije implementirana/izaziva pogrešku kako bi se vidjelo na koji način sustav reagira kada nešto nije u potpunosti ostvareno. Ispitni slučaj se treba sastojati od ulaza (npr. korisničko ime i lozinka), očekivanog izlaza ili rezultata, koraka ispitivanja i dobivenog izlaza ili rezultata.\\ }
			 
			 \textit{Izradu ispitnih slučajeva pomoću radnog okvira Selenium moguće je provesti pomoću jednog od sljedeća dva alata:}
			 \begin{itemize}
			 	\item \textit{dodatak za preglednik \textbf{Selenium IDE} - snimanje korisnikovih akcija radi automatskog ponavljanja ispita	}
			 	\item \textit{\textbf{Selenium WebDriver} - podrška za pisanje ispita u jezicima Java, C\#, PHP koristeći posebno programsko sučelje.}
			 \end{itemize}
		 	\textit{Detalji o korištenju alata Selenium bit će prikazani na posebnom predavanju tijekom semestra.}
			
			\eject 
		
		 
		\section{Dijagram razmještaja}
		
		U ovom dijagramu razmještaja opisana je programska potpora i topologija sklopov-
		lja korištena u radnom okruženju sustava. Sustav je izveden u arhitekturi ”klijent-
		poslužitelj” i podijeljen je na dva dijela, na poslužiteljsko i korisničko računalo.
		Baza podataka i web poslužitelj nalaze se na poslužiteljskom računalu, klijent se
		koristi web preglednikom kako bi pristupio web aplikaciji, a sva komunikacija
		izmedu klijenta i poslužitelja odvija se HTTPS vezom.
			
		\begin{figure}[H]
			\includegraphics[scale=0.8]{slike/Dijagram razmještaja.PNG} %veličina slike u odnosu na originalnu datoteku i pozicija slike
			\centering
			\caption{Specifikacijski dijagram razmještaja}
			\label{fig:promjene}
		\end{figure}
		
		\section{Upute za puštanje u pogon}
		
			\textbf{\textit{dio 2. revizije}}\\
		
			 \textit{U ovom poglavlju potrebno je dati upute za puštanje u pogon (engl. deployment) ostvarene aplikacije. Na primjer, za web aplikacije, opisati postupak kojim se od izvornog kôda dolazi do potpuno postavljene baze podataka i poslužitelja koji odgovara na upite korisnika. Za mobilnu aplikaciju, postupak kojim se aplikacija izgradi, te postavi na neku od trgovina. Za stolnu (engl. desktop) aplikaciju, postupak kojim se aplikacija instalira na računalo. Ukoliko mobilne i stolne aplikacije komuniciraju s poslužiteljem i/ili bazom podataka, opisati i postupak njihovog postavljanja. Pri izradi uputa preporučuje se \textbf{naglasiti korake instalacije uporabom natuknica} te koristiti što je više moguće \textbf{slike ekrana} (engl. screenshots) kako bi upute bile jasne i jednostavne za slijediti.}
			
			
			 \textit{Dovršenu aplikaciju potrebno je pokrenuti na javno dostupnom poslužitelju. Studentima se preporuča korištenje neke od sljedećih besplatnih usluga: \href{https://aws.amazon.com/}{Amazon AWS}, \href{https://azure.microsoft.com/en-us/}{Microsoft Azure} ili \href{https://www.heroku.com/}{Heroku}. Mobilne aplikacije trebaju biti objavljene na F-Droid, Google Play ili Amazon App trgovini.}
			
			
			\eject 